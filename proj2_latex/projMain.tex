
% This LaTeX was auto-generated from MATLAB code.
% To make changes, update the MATLAB code and republish this document.

\documentclass{article}
\usepackage{graphicx}
\usepackage{color}

\sloppy
\definecolor{lightgray}{gray}{0.5}
\setlength{\parindent}{0pt}

\begin{document}

    
    \begin{verbatim}
%------------------------------------------------------------------
%  ECE 446 Computer Project 1 Fall 2014
%  Project Done by: M. Iftekhar Tanveer (itanveer@cs.rochester.edu)
%  This project was coded and tested in MATLAB R2014b
%  Please run the scripts in MATLAB 2014b for correct execution
%------------------------------------------------------------------
%
% Part 1: Plot spectrums of noises
%
function projMain
clc;clear;
Fs = 20; % Given in question. Mhz is assumed

% Generate the noise signal
Noise = .05*(rand(512000,1)-.5);

% Estimate Spectrum
NoiseSpectrum2048 = EstimateSpectrum(Noise,1024,2048);

% Plot 64 Samples from the Spectrum
figure(1);
N_x = 64;
sftSpec = fftshift(NoiseSpectrum2048);
N = length(sftSpec); % Number of samples
N_2 = floor(N/2); % Center index
sftSpec = sftSpec(N_2-N_x:N_2+N_x-1); %Shift
% Plotting on log scale for better representation
plot(sftSpec);
ylabel('Spectral Estimate');
xlabel('Samples');
axis ([0,length(sftSpec),-0.0005,0.0005]);

% Additional spectrums of noise.
NoiseSpectrum1024 = EstimateSpectrum(Noise,512,1024);
NoiseSpectrum512 = EstimateSpectrum(Noise,256,512);
NoiseSpectrum256 = EstimateSpectrum(Noise,128,256);

% Plot all the Spectrums
figure(2);
plotSpectrums({NoiseSpectrum2048,NoiseSpectrum1024,NoiseSpectrum512, ...
    NoiseSpectrum256},Fs,{'NoiseSpectrum2048','NoiseSpectrum1024', ...
    'NoiseSpectrum512','NoiseSpectrum256'});

% -------------------------------------------------------------------
% Part 2: Noise passed through a LTI system: y[n] = x[n] - 0.9*x[n-1]
% -------------------------------------------------------------------
Rp = ShapeRandomProcess(Noise);
RpSpectrum2048=EstimateSpectrum(Rp,1024,2048);
RpSpectrum1024=EstimateSpectrum(Rp,512,1024);
RpSpectrum512=EstimateSpectrum(Rp,256,512);
RpSpectrum256=EstimateSpectrum(Rp,128,256);

figure(3);
plotSpectrums({RpSpectrum2048,RpSpectrum1024,RpSpectrum512, ...
    RpSpectrum256},Fs,{'NoiseSpectrum2048','NoiseSpectrum1024', ...
    'NoiseSpectrum512','NoiseSpectrum256'});
title('Spectrum of a White Noise after transformed by y[n] = x[n] - 0.9*x[n-1]')
axis([-10,10,0,max(RpSpectrum2048)]);

% -------------------------------------------------------------------
% Part 3: Analysis on Ashokan Farewell.wav
% -------------------------------------------------------------------
clc;clear;
figure(4);
[AF, Fs] = audioread('Ashokan Farewell.wav');
subplot(311)
PlotSpectra(AF(:,1),8192,16384,10);
title({'Spectograph of Ashokan Farewell','Skip = 8192, FFTsize = 16384'})
subplot(312)
PlotSpectra(AF(:,1),2048,4096,10);
title({'Spectograph of Ashokan Farewell','Skip = 2048, FFTsize = 4096'})
subplot(313)
PlotSpectra(AF(:,1),512,256,10);
title({'Spectograph of Ashokan Farewell','Skip = 512, FFTsize = 256'})
end
% -------------------------------------------------------------------
% Part 4: Relationship between Spectogram and Music
% -------------------------------------------------------------------
function player=runAF(skip,fftSize)
global player
AF=audioread('Ashokan Farewell.wav');
player = audioplayer(AF, 44100);
set(player,'TimerFcn','timerCallback','TimerPeriod',1);
Spectra =PlotSpectra(AF(:,1),skip,fftSize,10);
hold on;
play(player);
end
function timerCallback()
global player
nSample=get(player,'currentSample');
plot(floor(nSample/44100)+1,1,'*w','LineWidth',8);
end

% ========================= Helper functions =============================
function Spectrum = EstimateSpectrum(x,skip,FFTsize)
if(size(x,1)<size(x,2))
    x=x';
end
N = length(x); % Length of the signal
N_Sec = floor(N/skip); % Total number of sections

% Cropping out the sections, calculating PSD
% and taking average of the PSD's
X_n = zeros(FFTsize,1);
for n = 1:N_Sec-1
    % Minimizing memory usage by not saving each section
    X_n = X_n + (abs(fft(x((1:FFTsize) + (n-1)*skip))).^2)/FFTsize;
end
Spectrum = X_n/N_Sec;
end

function Rp=ShapeRandomProcess(N)
if(size(N,1)<size(N,2))
    N=N';
end
Rp = N - 0.9*[0;N(1:end-1)];
end

% MaxFreq is in KHz
function Spectra =PlotSpectra(x,skip,FFTsize,MaxFreq)
if(size(x,1)<size(x,2))
    x=x';
end
Fs = 44100/1024;
maxF_idx = floor(FFTsize*MaxFreq/Fs);

n_win = floor(length(x)/44100); % number of time windows
Spectra = [];
for idx = 1:n_win
    spec = EstimateSpectrum(x((1:(Fs*1024)) + (idx-1)*(Fs*1024)),skip,FFTsize);
    Spectra = [Spectra,spec(1:maxF_idx)];
end
Spectra = log(Spectra);
x = 0:(n_win-1);
y = (0:(maxF_idx-1))*Fs/FFTsize;
imagesc(x,y,Spectra);
colormap('Jet');
colorbar;
axis xy
ax = gca();
ax.XTick = min(x):25:max(x);
xlabel('Time (Sec)');
ylabel('Frequency (KHz)');
end

function plotSpectrums(Specs,Fs,LegendText)
for idx=1:length(Specs)
    N = length(Specs{idx}); % Number of samples
    f = (Fs/N)*((0:N-1) - ceil(N/2))'; % Freq axis

    % Shift the signal
    specSft = fftshift(Specs{idx});

    % Plotting on log scale for better representation
    plot(f,specSft);
    hold on
end
hold off

ylabel('Spectral Estimate');
xlabel('Frequency (MHz)');
axis ([min(f),max(f),-0.0005,0.0005]);
legend(LegendText);

end
\end{verbatim}

\includegraphics [width=4in]{projMain_01.eps}

\includegraphics [width=4in]{projMain_02.eps}

\includegraphics [width=4in]{projMain_03.eps}

\includegraphics [width=4in]{projMain_04.eps}



\end{document}
    
